\documentclass{assignment}
\usepackage{amsmath}
\usepackage{lipsum}
\usepackage{multicol}
\usepackage{fancyhdr}
\usepackage{graphicx}
\usepackage{blindtext}
\usepackage[dvipsnames]{xcolor}
\usepackage{enumitem}
\usepackage{cleveref}
\usepackage{color,soul}
\usepackage{amsfonts}
\def\code#1{\texttt{#1}}
\newtheorem{anm}{Anm}

\begin{document}


\assignmentTitle
{Lucas Frykman}{n/a}
{n/a}
{SF1930}
{assets/KTH_logga.png}
{Statistisk inlärning och dataanalys}
{Projekt}

\section{Introduktion}

Vi betraktar en modell av en tvättmaskin enligt angiven figur. 
Det finns en klump med våta kläder med massan $m$ inuti maskinen som skapar en obalans när maskinen roterar. 
Massan av maskinens roterande del utan kläder är $M$ och lasttrummans radie är $r$. 
Rotationsdelens rörelse styrs och dämpas av ett system som kan modelleras med fjädrar och dämpare enligt figuren, där $k$ och $c$ är fjäderkonstanten respektive dämpningskonstanten.


\section{Kod}
\lstinputlisting[language=Matlab,caption=foo]{assets/tmp.m} 
\end{document}
